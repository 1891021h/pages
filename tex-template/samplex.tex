%!TEX encoding = UTF-8
% ビルドレシピは「upLaTeX」を選択する
\documentclass[uplatex]{jsreport}
\renewcommand{\prechaptername}{第}
\renewcommand{\postchaptername}{章}
\usepackage{amsmath}
\usepackage{amssymb}
\usepackage{amsfonts}
\usepackage[top=30truemm,bottom=30truemm,left=25truemm,right=25truemm]{geometry}

\begin{document}
  \chapter{数列の極限}
    \section*{練習問題1}
    $$c_n=\frac{1}{\sqrt{n}}$$
    で与えられる数列$\{ c_n \}$が0に収束することを示せ。\\
    また,3つの数列$\{ a_n\},\{b_n\},\{c_n\}$が0に収束する速さを比較しなさい。
    ただし
    $$
    a_n = \frac{1}{n}
    $$
    $$
    b_n = \frac{1}{n^2}
    $$
    とする。。
  \chapter{関数の極限}
    \section{練習問題2}
    $ f(x)=\sqrt{x} $のとき,$ \displaystyle \lim_{x \to 1}{f(x)}=1$であることを示せ。
    \section{練習問題3}
    関数$ f(x)=x^2-2x $が$ \mathbb{R} $上の任意の点$a$で連続であることを示せ。
  \chapter{集合の上限と下限}
    \section{練習問題1’}
    集合$ S= \displaystyle \left\{ \frac{n}{n+1} \mid n=1,2,3,\cdots \right\}$ の上限と下限を求めよ(定義をみたすことを示せ)。
    \section{練習問題2’}
    集合
    $$ S_1=\left\{x \mid x>0 \right\} $$
    $$ S_2=\left\{x \mid x \leq 1 \right\} $$
    $$ S_3=\left\{x \mid 0<x \leq 1 \right\} $$
    について,$S_1$が下に有界,$S_2$が上に有界,$S_3$が有界であることを定義にしたがって確かめよ。
    \section{練習問題3’}
    実数$ \mathbb{R} $上の集合$S$は下に有界であるとする。$S$の最大下界が存在すれば,$S$の下限が存在し,$S$の最大下界と下限は一致することを示せ。
  \chapter{技術的な話}
    \section{練習問題5}
    $ \displaystyle \lim_{n \to \infty}{a_n} = \alpha $のとき,
    $ \displaystyle \lim_{n \to \infty}{ca_n} = c \alpha $
    が成り立つことを示せ。
    \section{練習問題6}
    $ \displaystyle \lim_{n \to \infty}{a_n} = \alpha \neq 0$のとき,$ a_n=0 $のとなる$ a_n $は有限個(0個の場合も含む)であることを示しなさい。
    \section{練習問題7}
    $ \displaystyle \lim_{n \to \infty}{a_n} = \alpha \neq 0$のとき,次が成り立つことを示しなさい。\\
    \\
    (1) $\exists N_1 \in \mathbb{N} \, $ s.t.$\forall n \geq N_1$, \, $ \displaystyle \frac{1}{|a_n|}<\frac{2}{|\alpha|}$ \\
    (2) $\displaystyle \lim_{n \to \infty}{\frac{1}{a_n}}=\frac{1}{\alpha}$
    \section{練習問題8}
    以下の(1)と(2)が成り立つことを示せ。
    \\
    (1) $ \displaystyle \lim_{x \to a}{f(x)} = \alpha$のとき,ある正の数$M$が存在して次が成り立つ;ある正の数$\rho$が存在して,$0<|x-a|<\rho$をみたすすべての$x$に対して,$f(x)<M$である。\\
    (2) $ \displaystyle \lim_{x \to a}{f(x)} = \alpha \, , \displaystyle \lim_{x \to a}{g(x)} = \beta$ならば,$ \displaystyle \lim_{x \to a}{f(x)g(x)} = \alpha \beta $
    \section{練習問題9}
    次の条件(A'),(B')を満たす数列の例をそれぞれ考えよ。\\
    (A') ある自然数$n$に対して,ある正の数$M$が存在して,$|a_n| \leq M$が成り立つ。\\
    (B') ある正の数$M$が存在して,$|a_n|>M$が成り立つ。
    \section{練習問題10}
    次の主張を満たす関数$f(x)$の例を挙げよ。また,もとの主張の否定形をつくり,それをみたす関数$f(x)$の例を挙げよ。\\
    (i) ある正の数$K$が存在して,任意の正の数$x$に対して,$f(x)\leq K$が成り立つ。\\
    (ii) 任意の正の数$K$に対して,$x<1$を満たすある正の数$x=x(K)$が存在して,$f(x)>K$が成り立つ。
\end{document}